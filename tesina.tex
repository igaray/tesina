\documentclass[10pt, a4paper]{report}
\usepackage[utf8]{inputenc}
\usepackage{enumitem}
\usepackage{hyperref}
\hypersetup{
    colorlinks=true,
    linkcolor=blue,
    filecolor=magenta,
    urlcolor=cyan,
}

\title{Tesina}
\author{Iñaki Garay}
\date{Septiembre 2020}

\begin{document}

\begin{titlepage}
\maketitle
\end{titlepage}

\tableofcontents

\section*{Introducción / Motivación}

\begin{itemize}[noitemsep]
\item Los compiladores tradicionales tienen una arquitectura de pipeline.
\item Los editores modernos usan LSP (Language Server Protocol).
\item Una implementacion de LSP require de componentes de compiladores (especialmente analisis).
\item La arquitectura de pipeline no se adapta bien a estos requerimientos modernos de reducir tiempos de compilacion y de proveer informacion online durante edicion.
\item Estos objetivos puede ser logrados mediante compilacion incremental.
\item La compilacion incremental puede ser lograda mediante una arquitectura basada en queries.
\item La arquitectura basada en queries es implementada tomando inspiracion de build systems.
\item Rust-analyzer es la segunda implementacion de LSP para rust.
\item Porque no funciono la primera iteracion?
\item Rust-Analyzer usa una libreria, Salsa, para cachear las queries parciales.
\item Como funciona Salsa?
\end{itemize}

\section*{Metas vs Mecanismo}

\subsection*{Arquitectura Basada en Consultas}

\subsection*{Caso de Estudio: Rustc}

\subsection*{Caso de Estudio: Rust-analyzer}

\subsection*{Salsa}

\section*{Fuentes y Referencias}

\begin{itemize}[noitemsep]
\item \href{https://www.youtube.com/watch?v=wSdV1M7n4gQ}{Youtube: Anders Hejlsberg on Modern Compiler Construction}
\item \href{https://en.wikipedia.org/wiki/Incremental_compiler}{Wikipedia: Incremental Compiler}
\item \href{https://www.microsoft.com/en-us/research/publication/build-systems-la-carte/}{Build Systems A La Carte}
\item \href{https://www.youtube.com/watch?v=b_T-eCToX1I}{Youtube: 2016 LLVM Developers’ Meeting: D. Dunbar “A New Architecture for Building Software”}
\item \href{https://dl.acm.org/doi/10.1145/502949.502889}{An approach to incremental compilation}
\item \href{https://ollef.github.io/blog/posts/query-based-compilers.html}{Olle Fredriksson: Query-based compiler architectures}
\item \href{https://blog.rust-lang.org/2016/09/08/incremental.html}{Rust Blog: Incremental Compilation}
\item Rustc Dev Guide
	\begin{itemize}[noitemsep]
	\item \href{https://rustc-dev-guide.rust-lang.org/overview.html}{Overview of the Compiler}
	\item \href{https://rustc-dev-guide.rust-lang.org/compiler-src.html}{High-level overview of the compiler source}
	\item \href{https://rustc-dev-guide.rust-lang.org/query.html}{Queries: demand-driven compilation}
		\begin{itemize}[noitemsep]
		\item \href{https://rustc-dev-guide.rust-lang.org/queries/query-evaluation-model-in-detail.html}{The Query Evaluation Model in Detail}
		\item \href{https://rustc-dev-guide.rust-lang.org/queries/incremental-compilation.html}{Incremental compilation}
		\item \href{https://rustc-dev-guide.rust-lang.org/queries/incremental-compilation-in-detail.html}{Incremental Compilation In Detail}
		\item \href{https://rustc-dev-guide.rust-lang.org/incrcomp-debugging.html}{Debugging and Testing Dependencies}
		\item \href{https://rustc-dev-guide.rust-lang.org/queries/profiling.html}{Profiling Queries}
		\item \href{https://rustc-dev-guide.rust-lang.org/salsa.html}{How Salsa works}
		\end{itemize}
	\item \href{}{}
	\end{itemize}
\item Rust Analyzer
	\begin{itemize}[noitemsep]
	\item \href{https://rust-analyzer.github.io/}{Rust Analyzer}
	\item \href{https://rust-analyzer.github.io/manual.html}{Manual}
	\item \href{https://rust-analyzer.github.io/blog}{Blog}
	\item \href{https://youtu.be/ANKBNiSWyfc}{Youtube: Rust-Analyzer Guide}
	\item \href{https://github.com/rust-analyzer/rust-analyzer/tree/master/docs/dev}{rust-analyzer/tree/master/docs/dev}
	\item \href{https://ferrous-systems.com/blog/rust-analyzer-2019/}{Rust Analyzer in 2018 and 2019}
	\item \href{https://ferrous-systems.com/blog/rust-analyzer-status-opencollective/}{Status of rust-analyzer}
	\item \href{https://www.youtube.com/playlist?list=PLXajQV_H-DxLMBt0amcuxgTeOTj6L-YGl}{Youtube: Rust Analyzer Q\&A}
	\item \href{https://blog.logrocket.com/intro-to-rust-analyzer/}{2020 Intro to Rust Analyzer}
	\item \href{https://dev.to/bnjjj/what-i-learned-contributing-to-rust-analyzer-4c7e}{2020 What I learned contributing to Rust-Analyzer}
	\end{itemize}
\item \href{https://salsa-rs.github.io/salsa/}{The Salsa Book}
\item \href{https://www.youtube.com/playlist?list=PL85XCvVPmGQh0P_VEPVM2ZIlBwl4MQMNY}{Youtube: Incremental Compilation Working Group}
\item \href{https://www.youtube.com/watch?v=N6b44kMS6OM}{Youtube: Responsive compilers - Nicholas Matsakis - PLISS 2019}
\item \href{https://www.youtube.com/watch?v=LIYkT3p5gTs}{Youtube: Things I Learned (TIL) - Nicholas Matsakis - PLISS 2019}
\item \href{https://www.youtube.com/watch?v=_muY4HjSqVw}{Youtube: How Salsa Works (2019.01)}
\item \href{https://www.youtube.com/watch?v=i_IhACacPRY}{Salsa In More Depth (2019.01)}
\item \href{https://www.youtube.com/watch?v=Xr-rBqLr-G4}{RLS 2.0, Salsa, and Name Resolution}
\end{itemize}

\end{document}
