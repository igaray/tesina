\documentclass[12pt, a4paper]{report}
\usepackage[utf8]{inputenc}
\usepackage{enumitem}
\usepackage{hyperref}
\hypersetup{
    colorlinks=true,
    linkcolor=blue,
    filecolor=magenta,
    urlcolor=cyan,
}
\usepackage{wasysym}

\title{Tesina}
\author{Iñaki Garay}
\date{Septiembre 2020}

\begin{document}

\begin{titlepage}
\maketitle
\end{titlepage}

\tableofcontents

\section*{Introducción / Esquema General}

\begin{itemize}[noitemsep]
\item Los compiladores tradicionales tienen una arquitectura de pipeline.
\item Los editores y entornos de desarrollo modernos usan LSP (Language Server Protocol). Porque?
\item Una implementacion de LSP require de componentes de compiladores (especialmente analisis).
\item La arquitectura de pipeline no se adapta bien a estos requerimientos modernos de reducir tiempos de compilacion y de proveer informacion online durante edicion.
\item Estos objetivos puede ser logrados mediante compilacion incremental.
\item La compilacion incremental puede ser lograda mediante una arquitectura basada en queries.
\item La arquitectura basada en queries es implementada tomando inspiracion de build systems.
\item Rust-analyzer es la segunda implementacion de LSP para rust.
\item Porque no funciono la primera iteracion?
\item Rust-Analyzer usa una libreria, Salsa, para cachear las queries parciales.
\item Como funciona Salsa?
\end{itemize}

\section*{Motivaciones}

Los compiladores ya no son cajas negras que ingestan un conjunto de
archivos fuente y producen codigo ensamblador.
De compiladores modernos se espera que:

\begin{itemize}[noitemsep]
\item Sean incrementales, es decir, si se recompila el proyecto
despues de producir modificaciones en el código fuente, solo se
recompile lo que fue afectado por esas modificaciones.
\item Provean funcionalidad para editores, e.g. saltar a definición,
encontrar el tipo de una expresión en una ubicación dada, y mostrar
errores al editar.
\end{itemize}

Una arquitectura de compilador tradicional se

\noindent
\texttt{GRAFICO}

\begin{verbatim}
- We push source text down a pipeline and run a fixed set of
  transformations until we finally output assembly code or some other
  target language. Along the way we often need to read and update some
  state. For example, we might update a type table during type
  checking so we can later look up the type of entities that the code
  refers to.
\end{verbatim}

\subsection*{Language Server Protocol}

\begin{verbatim}
The Language Server Protocol (LSP) is an open, JSON-RPC-based protocol
for use between source code editors or integrated development
environments (IDEs) and servers that provide programming
language-specific features. The goal of the protocol is to allow
programming language support to be implemented and distributed
independently of any given editor or IDE.
\end{verbatim}

\begin{verbatim}
Adding features like auto complete, go to definition, or documentation
on hover for a programming language takes significant effort.
Traditionally this work had to be repeated for each development tool,
as each tool provides different APIs for implementing the same
feature.

A Language Server is meant to provide the language-specific smarts
and communicate with development tools over a protocol that enables
inter-process communication.

The idea behind the Language Server Protocol (LSP) is to standardize
the protocol for how such servers and development tools communicate.
This way, a single Language Server can be re-used in multiple
development tools, which in turn can support multiple languages with
minimal effort.

LSP is a win for both language providers and tooling vendors!
\end{verbatim}

\begin{verbatim}
Modern integrated development environments (IDEs) provide developers
with sophisticated features like code completion, refactoring,
navigating to a symbol's definition, syntax highlighting, and error
and warning markers.

For example, in a text-based programming language, a programmer might
want to rename a method read. The programmer could either manually
edit the respective source code files and change the appropriate
occurrences of the old method name into the new name, or instead use
an IDE's refactoring capabilities to make all the necessary changes
automatically. To be able to support this style of refactoring, an
IDE needs a sophisticated understanding of the programming language
that the program's source is written in. A programming tool without
such an understanding—for example, one that performs a naive
search-and-replace instead—could introduce errors. When renaming a
read method, for example, the tool should not replace the partial
match in a variable that might be called readyState, nor should it
replace the portion of a code comment containing the word "already".
Neither should renaming a local variable read, for example, end up
altering similarly named variables in other scopes.

Conventional compilers or interpreters for a specific programming
language are typically unable to provide these language services,
because they are written with the goal of either transforming the
source code into object code or immediately executing the code.
Additionally, language services must be able to handle source code
that is not well-formed, e.g. because the programmer is in the middle
of editing and has not yet finished typing a statement, procedure, or
other construct. Additionally, small changes to a source code file
which are done during typing usually change the semantics of the
program. In order to provide instant feedback to the user, the editing
tool must be able to very quickly evaluate the syntactical and
semantical consequences of a specific modification. Compilers and
interpreters therefore provide a poor candidate for producing the
information needed for an editing tool to consume.

Prior to the design and implementation of the Language Server Protocol
for the development of Visual Studio Code, most language services were
generally tied to a given IDE or other editor. In the absence of the
Language Server Protocol, language services are typically implemented
by utilizing a tool-specific extension API. Providing the same
language service to another editing tool requires effort to adapt the
existing code so that the service may target the second editor's
extension interfaces.

The Language Server Protocol allows for decoupling language services
from the editor so that the services may be contained within a general
purpose language server. Any editor can inherit sophisticated support
for many different languages by making use of existing language
servers. Similarly, a programmer involved with the development of a
new programming language can make services for that language available
to existing editing tools. Making use of language servers via the
Language Server Protocol thus also reduces the burden on vendors of
editing tools, because vendors do not need to develop language
services of their own for the languages the vendor intends to support,
as long as the language servers have already been implemented.
The Language Server Protocol also enables the distribution and
development of servers contributed by an interested third-party, such
as end users, without additional involvement by either the vendor of
the compiler for the programming language in use or the vendor of the
editor to which the language support is being added.

LSP is not restricted to programming languages. It can be used for any
kind of text-based language, like specifications or domain-specific
languages (DSL).
\end{verbatim}

\subsection*{Velocidad de Compilación}

\begin{verbatim}
- Improving compile times has actually been a major development focus after Rust
  reached 1.0 -- although, up to this point, much of the work towards this goal
  has gone into laying architectural foundations within the compiler and we are
  only slowly beginning to see actual results.
- One of the projects that is building on these foundations, and that should
  help improve compile times a lot for typical workflows, is incremental
  compilation.
  Incremental compilation avoids redoing work when you recompile a crate, which
  will ultimately lead to a much faster edit-compile-debug cycle.
\end{verbatim}

\section*{Mecanismos}

\subsection*{Compilación Incremental}

La compilación incremental es una forma de computación incremental aplicada a la compilación.
En contraste con compiladores comúnes que realizan "builds limpios" y ante un cambio en el código fuente recompilan todas las unidades de compilación, un compilador incremental solo recompila las unidades modificadas.
Al construir sobre el trabajo hecho previamente, el compilador incremental evita la ineficiencia de repetir trabajo ya realizado.
Se puede decir que un compilador incremental reduce la granularidad de las unidades de compilación tradicionales a la vez que mantiene la semántica del lenguaje.

Muchas herramientas de desarrollo aprovechan compiladores incrementales para proveer a sus usuarios un entorno mucho mas interactivo.
No es inusual que un compilador incremental sea invocado por cada cambio en un archivo fuente, de tal manera que el usuario es informado inmediatamente de cualquier error de compilación causado por sus modificaciones.
Este esquema, en contraste con el modelo de compilación tradicional, acorta el ciclo de desarrollo considerablemente.

Una desventaja de este esquema es que el compilador no puede optimizar fácilmente el código que compila, dada la localidad y el alcance reducido de los cambios.
Normalmente esto no es un problema, dado que la optimización del código generado se aplica solamente al producir un \textit{release build}, instancia en la cual se puede usar el compilador tradicional.

\subsection*{Arquitectura Basada en Consultas}

\begin{verbatim}
- Going from pipeline to queries

- What does it take to get the type of a qualified name?

- In a pipeline-based architecture we would just look it up in the
  type table.

- With queries, we have to think differently.

- Instead of relying on having updated some piece of state, we do it
  as if it was done from scratch.

- As a first iteration, we do it completely from scratch

- We first find out what file the name comes from, which might be
  Data/List.vix for Data.List, then read the contents of the file,
  parse it, perhaps we do name resolution to find out what the names
  in the code refer to given what is imported, and last we look up the
  name-resolved definition and type check it, returning its type.

-
```
fetchType :: QualifiedName -> IO Type
fetchType (QualifiedName moduleName name) = do
fileName <- moduleFileName moduleName
sourceCode <- readFile fileName
parsedModule <- parseModule sourceCode
resolvedModule <- resolveNames parsedModule
let definition = lookup name resolvedModule
inferDefinitionType definition
```

- Let's first refactor the code into smaller functions:

- Note that each of the functions do everything from scratch on their
  own, i.e. they're each doing a (longer and longer) prefix of the
  work you'd do in a pipeline.

- I've found this to be a common pattern in my query-based compilers.

- One way to make this efficient would be to add a memoisation layer
  around each function.

- That way, we do some expensive work the first time we invoke a
  function with a specific argument, but subsequent calls are cheap as
  they can return the cached result.

- This is essentially what we'll do, but we won't use a separate cache
  per function, but instead have a central cache, indexed by the
  query.
\end{verbatim}

\begin{verbatim}
- **Why Incremental Compilation in the First Place?**
- Much of a programmer's time is spent in an edit-compile-debug workflow:
	- you make a small change (often in a single module or even function),
	- you let the compiler translate the code into a binary, and finally
	- you run the program or a bunch of unit tests in order to see results
 of the change.
- After that it's back to step one, making the next small change informed by
the knowledge gained in the previous iteration.
This essential feedback loop is at the core of our daily work.
We want the time being stalled while waiting for the compiler to produce
an executable program to be as short as possible.
- Incremental compilation is a way of exploiting the fact that little
changes between compiles during the regular programming workflow:
Many, if not most, of the changes done in between two compilation sessions
only have local impact on the machine code in the output binary, while the
rest of the program, same as at the source level, will end up exactly the
same, bit for bit.
Incremental compilation aims at retaining as much of those unchanged parts
as possible while redoing only that amount of work that actually must be
redone.
\end{verbatim}

\begin{verbatim}
- **How Do You Make Something "Incremental"?**

- We have already heard that computing something incrementally means
  updating only those parts of the computation's output that need to be
  adapted in response to a given change in the computation's inputs.
- One basic strategy we can employ to achieve this is to view one big
  computation (like compiling a program) as a composite of many smaller,
  interrelated computations that build up on each other.
- Each of those smaller computations will yield an intermediate result
  that can be cached and hopefully re-used in a later iteration, sparing
  us the need to re-compute that particular intermediate result again.
\end{verbatim}

\begin{verbatim}
- **An Incremental Compiler**

- The way we chose to implement incrementality in the Rust compiler is
  straightforward: An incremental compilation session follows exactly
  the same steps in the same order as a batch compilation session.
- However, when control flow reaches a point where it is about to
  compute some non-trivial intermediate result, it will try to load that
  result from the incremental compilation cache on disk instead.
- If there is a valid entry in the cache, the compiler can just skip
  computing that particular piece of data. Let's take a look at a
  (simplified) overview of the different compilation phases and the
  intermediate results they produce:
- First the compiler will parse the source code into an abstract syntax
  tree (AST).
  The AST then goes through the analysis phase which produces type
  information and the MIR for each function.
  After that, if analysis did not find any errors, the codegen phase
  will transform the MIR version of the program into its machine code
  version, producing one object file per source-level module.
  In the last step all the object files get linked together into the
  final output binary which may be a library or an executable.
- So, this seems pretty simple so far: Instead of computing something a
  second time, just load the value from the cache.
  Things get tricky though when we need to find out if it's actually
  valid to use a value from the cache or if we have to re-compute it
  because of some changed input.
\end{verbatim}

\subsection*{Memoizacion y Seguimiento de Dependencias}

\begin{verbatim}
- Rock, Shake, Salsa
- This functionality is provided by Rock, a library that packages up
  some functionality for creating query-based compilers.
- Rock is an experimental library heavily inspired by Shake and the
  Build systems à la carte paper.
- It essentially implements a build system framework, like make.
- Build systems have a lot in common with modern compilers since we want
  them to be incremental, i.e. to take advantage of previous build
  results when building anew with few changes.
- But there's also a difference: Most build systems don't care about the
  types of their queries since they work at the level of files and file
  systems.
- Build systems à la carte is closer to what we want.
- There the user writes a bunch of computations, tasks, choosing a
  suitable type for keys and a type for values.
- The tasks are formulated assuming they're run in an environment where
  there is a function fetch of type Key -> Task Value, where Task is a
  type for describing build system rules, that can be used to fetch the
  value of a dependency with a specific key.
- In our above example, the key type might look like this:
- The build system has control over what code runs when we do a fetch,
  so by varying that it can do fine-grained dependency tracking,
  memoisation, and incremental updates.
- Build systems à la carte is also about exploring what kind of build
  systems we get when we vary what Task is allowed to do, e.g. if it's a
  Monad or Applicative.
- In Rock, we're not exploring that, so our Task is a thin layer on top
  of IO.
- A problem that pops up now, however, is that there's no satisfactory
  type for Value.
- We want fetch (ParsedModuleKey "Data.List") to return a ParsedModule,
  while fetch (TypeKey "Data.List.map") should return something of type
  Type.
\end{verbatim}

\begin{verbatim}
- Indexed queries

- Rock allows us to index the key type by the return type of the query.
  The Key type in our running example becomes the following GADT:
	```
	data Key a where
	  ParsedModuleKey :: ModuleName -> Key ParsedModule
	  ResolvedModuleKey :: ModuleName -> Key ResolvedModule
	  TypeKey :: QualifiedName -> Key Type
	```
- The fetch function gets the type forall a. Key a -> Task a, so we get
  a ParsedModule when we run fetch (ParsedModuleKey "Data.List"), like
  we wanted, because the return type depends on the key we use.
- Now that we know what fetch should look like, it's also worth
  revealing what the Task type looks like in Rock, more concretely.
- As mentioned, it's a thin layer around IO, providing a way to fetch
  keys (like Key above):
- The rules of our compiler, i.e. its "Makefile", then becomes the
  following function, reusing the functions from above:
	```
	rules :: Key a -> Task a
	rules key = case key of
	  ParsedModuleKey moduleName ->
	    fetchParsedModule moduleName
	  ResolvedModuleKey moduleName ->
	    fetchResolvedModule moduleName
	  TypeKey qualifiedName ->
	    fetchType qualifiedName
	```
\end{verbatim}

\begin{verbatim}
- Caching

- The most basic way to run a Task in Rock is to directly call the rules
  function when a Task fetches a key.
- This results in an inefficient build system that recomputes every
  query from scratch.
- But the Rock library lets us layer more functionality onto our rules
  function, and one thing that we can add is memoisation.
- If we do that Rock caches the result of each fetched key by storing
  the key-value pairs of already performed fetches in a dependent
  hashmap.
- This way, we perform each query at most once during a single run of
  the compiler.
\end{verbatim}

\begin{verbatim}
- Verifying dependencies and reusing state

- Another kind of functionality that can be layered onto the rules
  function is incremental updates. When it's used, Rock keeps track of
  what dependencies a task used when it was executed (much like Shake)
  in a table, i.e. what keys it fetched and what the values were.
- Using this information it's able to determine when it's safe to reuse
  the cache from a previous run of the compiler even though there might
  be changes in other parts of the dependency graph.
- This fine-grained dependency tracking also allows reusing the cache
  when a dependency of a task changes in a way that has no effect.
- For example, whitespace changes might trigger a re-parse, but since
  the AST is the same, the cache can be reused in queries that depend on
  the parse result.
\end{verbatim}

\begin{verbatim}
- Reverse dependency tracking

- Verifying dependencies can be too slow for real-time tooling like
  language servers, because large parts of the dependency graph have to
  be traversed just to check that most of it is unchanged even for tiny
  changes.
- For example, if we make changes to a source file with many large
  imports, we need to walk the dependency trees of all of the imports
  just to update the editor state for that single file.
- This is because dependency verification by itself needs to go all the
  way to the root queries for all the dependencies of a given query,
  which can often be a large proportion of the whole dependency tree.
- To fix this, Rock can also be made to track reverse dependencies
  between queries.
- When e.g. a language server detects that a single file has changed,
  the reverse dependency tree is used to invalidate the cache just for
  the queries that depend on that file by walking the reverse
  dependencies starting from the changed file.
- Since the imported modules don't depend on that file, they don't need
  to be re-checked, resulting in much snappier tooling!
\end{verbatim}

\begin{verbatim}
- **Dependency Graphs**

- There is a formal method that can be used to model a computation's
  intermediate results and their individual "up-to-dateness" in a
  straightforward way: dependency graphs.
- It looks like this: Each input and each intermediate result is
  represented as a node in a directed graph.
  The edges in the graph, on the other hand, represent which
  intermediate result or input can have an influence on some other
  intermediate result.
- Note, by the way, that the above graph is a tree just because the
  computation it models has the form of a tree.
  In general, dependency graphs are directed acyclic graphs
- What makes this data structure really useful is that we can ask it
  questions of the form "if X has changed, is Y still up-to-date?".
  We just take the node representing Y and collect all the inputs Y
  depends on by transitively following all edges originating from Y.
  If any of those inputs has changed, the value we have cached for Y
  cannot be relied on anymore.
\end{verbatim}

\begin{verbatim}
- **Dependency Tracking in the Compiler**

- When compiling in incremental mode, we always build the dependency
  graph of the produced data: every time, some piece of data is written
  (like an object file), we record which other pieces of data we are
  accessing while doing so.
- The emphasis is on recording here. At any point in time the compiler
  keeps track of which piece of data it is currently working on (it does
  so in the background in thread-local memory).
- This is the currently active node of the dependency graph.
  Conversely, the data that needs to be read to compute the value of the
  active node is also tracked: it usually already resides in some kind
  container (e.g. a hash table) that requires invoking a lookup method
  to access a specific entry.
- We make good use of this fact by making these lookup methods
  transparently create edges in the dependency graph: whenever an entry
  is accessed, we know that it is being read and we know what it is
  being read for (the currently active node).
- This gives us both ends of the dependency edge and we can simply add
  it to the graph.
  At the end of the compilation sessions we have all our data nicely
  linked up, mostly automatically.
- This dependency graph is then stored in the incremental compilation
  cache directory along with the cache entries it describes.
- At the beginning of a subsequent compilation session, we detect which
  inputs (=AST nodes) have changed by comparing them to the previous
  version.
  Given the graph and the set of changed inputs, we can easily find all
  cache entries that are not up-to-date anymore and just remove them
  from the cache.
- Anything that has survived this cache validation phase can safely be
  re-used during the current compilation session.
- There are a few benefits to the automated dependency tracking approach
  we are employing.
  Since it is built into the compiler's internal APIs, it will stay
  up-to-date with changes to the compiler, and it is hard to
  accidentally forget about.
  And if one still forgets using it correctly (e.g. by not declaring the
  correct active node in some place) then the result is an overly
  conservative, but still "correct" dependency graph:
  It will negatively impact the re-use ratio but it will not lead to
  incorrectly re-using some outdated piece of data.
- Another aspect is that the system does not try to predict or compute
  what the dependency graph is going to look like, it just keeps track.
  A large part of our (yet to be written) regression tests, on the other
  hand, will give a description of what the dependency graph for a given
  program ought to look like. This makes sure that the actual graph and
  the reference graph are arrived at via different methods, reducing the
  risk that both the compiler and the test case agree on the same, yet
  wrong, value.
\end{verbatim}

\begin{verbatim}
- **"Faster! Up to 15% or More."***

- Let's take a look at some of the implications of what we've learned so
  far:
	- The dependency graph reflects the actual dependencies between
	  parts of the source code and parts of the output binary.
	- If there is some input node that is reachable from many
	  intermediate results, e.g. a central data type that is used in
	  almost every function, then changing the definition of that data
	  type will mean that everything has to be compiled from scratch,
	  there's no way around it.
- In other words, the effectiveness of incremental compilation is very
  sensitive to the structure of the program being compiled and the
  change being made. Changing a single character in the source code
  might very well invalidate the whole incremental compilation cache.
  Usually though, this kind of change is a rare case and most of the
  time only a small portion of the program has to be recompiled.
\end{verbatim}

\subsection*{Caso de Estudio: Rustc}

Rustc, el compilador de rust, tiene su propia implementacion de queries.

\begin{verbatim}
- **The Current Status of the Implementation** (09/2019)

- For the first spike implementation of incremental compilation, what we
  call the alpha version now, we chose to focus on caching object files.
- Consequently, if this phase can be skipped at least for part of a code
  base, this is where the biggest impact on compile times can be
  achieved.
- With that in mind, we can also give an upper bound on how much time
  this initial version of incremental compilation can save:
  If the compiler spends X seconds optimizing when compiling your crate,
  then incremental compilation will reduce compile times at most by
  those X seconds.
- Another area that has a large influence on the actual effectiveness of
  the alpha version is dependency tracking granularity: It's up to us
  how fine-grained we make our dependency graphs, and the current
  implementation makes it rather coarse in places.
  For example, the dependency graph only knows a single node for all
  methods in an impl. As a consequence, the compiler will consider all
  methods of that impl as changed if just one of them is changed.
  This of course will mean that more code will be re-compiled than is
  strictly necessary.
\end{verbatim}

\subsection*{Caso de Estudio: Rust-analyzer y Salsa}

Rust-Analyzer utiliza una libreria llamada salsa.

\subsection*{Caso de Estudio: Como funciona salsa?}

La idea central de salsa es definir el programa como un conjunto de \textit{queries}.
Cada query se usa como una función $K \to V$ que mapea de una clave de tipo $K$ a un valor de tipo $V$.

Las queries en salsa son de dos variedades basicas:
\begin{itemize}[noitemsep]
\item \textbf{Entradas:}
definen los inputs basicos al sistema, los cuales pueden cambiar en cualquier momento.
\item \textbf{Funciones:}
funciones puras (sin efectos secundarios) que transforman las entradas en otros valores.
Los resultados de estas queries se memoizan para evitar recomputarlas.
Cuando se modifican las entradas, salsa determina cuales valores memoizados pueden ser reutilizados y cuales deben ser recomputados.
\end{itemize}

El esquema general de utilizacion de salsa consiste en tres pasos:

\begin{enumerate}[noitemsep]
\item Definir uno o mas grupos de queries que contendran las entradas y las queries requeridas.
Se puede definir mas de un grupo para separar las queries en componentes.
\item Definir las queries.
\item Definir la base de datos, la cual contendra el almacenamiento para las entradas y queries utilizadas.
\end{enumerate}

\section*{Conclusiones}

\begin{verbatim}
- Most modern languages need to have a strategy for tooling, and
  building compilers around query systems seems like an extremely
  promising approach to me.
- With queries the compiler writer doesn't have to handle updates to and
  invalidation of a bunch of ad-hoc caches, which can be the result when
  adding incremental updates to a traditional compiler pipeline.
- In a query-based system it's all handled centrally once and for all,
  which means there's less of a chance it's wrong.
- Queries are excellent for tooling because they allow us to ask for the
  value of any query at any time without worrying about order or
  temporal effects, just like a well-written Makefile.
- The system will compute or retrieve cached values for the query and
  its dependencies automatically in an incremental way.
- Query-based compilers are also surprisingly easy to parallelise.
- Since we're allowed to make any query at any time, and they're
  memoised the first time they're run, we can fire off queries in
  parallel without having to think much.
- In Sixty, the default behaviour is for all input modules to be type
  checked in parallel.
\end{verbatim}

\begin{verbatim}
- **Future Plans** (09/2019)
- The section on the current status already laid out the two major axes
  along which we will pursue increased efficiency:
	- Cache more intermediate results, like MIR and type information,
	  which will allow the compiler to skip more and more steps.
	- Make dependency tracking more precise, so that the compiler
	  encounters fewer false positives during cache invalidation.
\end{verbatim}

\section*{Fuentes y Referencias}

\begin{itemize}[noitemsep]
	
\item General
	\begin{itemize}[noitemsep]
	\item \href{https://www.youtube.com/watch?v=wSdV1M7n4gQ}{\CheckedBox Youtube: Anders Hejlsberg on Modern Compiler Construction}
	\item \href{https://en.wikipedia.org/wiki/Incremental_compiler}{\CheckedBox Wikipedia: Incremental Compiler}
	\item \href{https://ollef.github.io/blog/posts/query-based-compilers.html}{\CheckedBox Olle Fredriksson: Query-based compiler architectures}
	\item \href{https://blog.rust-lang.org/2016/09/08/incremental.html}{\CheckedBox Rust Blog: Incremental Compilation}
	\item \href{https://www.microsoft.com/en-us/research/publication/build-systems-la-carte/}{\Square Build Systems A La Carte}
	\item \href{https://www.youtube.com/watch?v=b_T-eCToX1I}{\Square Youtube: 2016 LLVM Developers’ Meeting: D. Dunbar “A New Architecture for Building Software”}
	\item \href{https://dl.acm.org/doi/10.1145/502949.502889}{\Square An approach to incremental compilation}
	\end{itemize}

\item Rustc Dev Guide
	\begin{itemize}[noitemsep]
	\item \href{https://rustc-dev-guide.rust-lang.org/overview.html}{\Square Overview of the Compiler}
	\item \href{https://rustc-dev-guide.rust-lang.org/compiler-src.html}{\Square High-level overview of the compiler source}
	\item \href{https://rustc-dev-guide.rust-lang.org/query.html}{\Square Queries: demand-driven compilation}
		\begin{itemize}[noitemsep]
		\item \href{https://rustc-dev-guide.rust-lang.org/queries/query-evaluation-model-in-detail.html}{\Square The Query Evaluation Model in Detail}
		\item \href{https://rustc-dev-guide.rust-lang.org/queries/incremental-compilation.html}{\Square Incremental compilation}
		\item \href{https://rustc-dev-guide.rust-lang.org/queries/incremental-compilation-in-detail.html}{\Square Incremental Compilation In Detail}
		\item \href{https://rustc-dev-guide.rust-lang.org/incrcomp-debugging.html}{\Square Debugging and Testing Dependencies}
		\item \href{https://rustc-dev-guide.rust-lang.org/queries/profiling.html}{\Square Profiling Queries}
		\item \href{https://rustc-dev-guide.rust-lang.org/salsa.html}{\Square How Salsa works}
		\end{itemize}
	\end{itemize}

\item Rust Analyzer
	\begin{itemize}[noitemsep]
	\item \href{https://rust-analyzer.github.io/}{\Square Rust Analyzer}
	\item \href{https://rust-analyzer.github.io/manual.html}{\Square Manual}
	\item \href{https://rust-analyzer.github.io/blog}{\Square Blog}
	\item \href{https://github.com/rust-analyzer/rust-analyzer/tree/master/docs/dev}{\Square rust-analyzer/tree/master/docs/dev}
	\item \href{https://ferrous-systems.com/blog/rust-analyzer-2019/}{\Square Rust Analyzer in 2018 and 2019}
	\item \href{https://ferrous-systems.com/blog/rust-analyzer-status-opencollective/}{\Square Status of rust-analyzer}
	\item \href{https://blog.logrocket.com/intro-to-rust-analyzer/}{\Square 2020 Intro to Rust Analyzer}
	\item \href{https://dev.to/bnjjj/what-i-learned-contributing-to-rust-analyzer-4c7e}{\Square 2020 What I learned contributing to Rust-Analyzer}
	\item \href{https://www.youtube.com/watch?v=7_7ckOKZCJE}{\Square Youtube: Are we *actually* IDE yet? A look on the Rust IDE Story - Igor Matuszewski}
	\item \href{https://www.youtube.com/watch?v=ANKBNiSWyfc}{\Square Youtube: Rust analyzer guide}
	\item \href{https://www.youtube.com/watch?v=DGAuLWdCCAI}{\Square Youtube: rust analyzer syntax trees}
	\item \href{https://www.youtube.com/watch?v=Lmp3P9WNL8o}{\Square Youtube: rust-analyzer type-checker overview by flodiebold}
	\item \href{https://www.youtube.com/playlist?list=PLXajQV_H-DxLMBt0amcuxgTeOTj6L-YGl}{\Square Youtube: Rust Analyzer Q\&A}
	\end{itemize}

\item Salsa
	\begin{itemize}[noitemsep]
	\item \href{https://salsa-rs.github.io/salsa/}{\Square The Salsa Book}
	\item \href{https://www.youtube.com/playlist?list=PL85XCvVPmGQh0P_VEPVM2ZIlBwl4MQMNY}{\Square Youtube: Incremental Compilation Working Group}
	\item \href{https://www.youtube.com/watch?v=N6b44kMS6OM}{\Square Youtube: Responsive compilers - Nicholas Matsakis - PLISS 2019}
	\item \href{https://www.youtube.com/watch?v=LIYkT3p5gTs}{\Square Youtube: Things I Learned (TIL) - Nicholas Matsakis - PLISS 2019}
	\item \href{https://www.youtube.com/watch?v=_muY4HjSqVw}{\Square Youtube: How Salsa Works (2019.01)}
	\item \href{https://www.youtube.com/watch?v=i_IhACacPRY}{\Square Youtube: Salsa In More Depth (2019.01)}
	\item \href{https://www.youtube.com/watch?v=Xr-rBqLr-G4}{\Square Youtube: RLS 2.0, Salsa, and Name Resolution}
	\end{itemize}

\item Rust Compilation Speed
	\begin{itemize}[noitemsep]
	\item \href{https://vfoley.xyz/rust-compile-speed-tips/}{\Square How to alleviate the pain of Rust compile times}
	\item \href{https://blog.mozilla.org/nnethercote/2016/10/14/how-to-speed-up-the-rust-compiler/}{\Square Nethercote: How to speed up the Rust compiler}
	\item \href{https://blog.mozilla.org/nnethercote/2016/11/23/how-to-speed-up-the-rust-compiler-some-more/}{\Square Nethercote: How to speed up the Rust compiler some more}
	\item \href{https://blog.mozilla.org/nnethercote/2018/04/30/how-to-speed-up-the-rust-compiler-in-2018/}{\Square Nethercote: How to speed up the Rust compiler in 2018}
	\item \href{https://blog.mozilla.org/nnethercote/2018/06/05/how-to-speed-up-the-rust-compiler-some-more-in-2018/}{\Square Nethercote: How to speed up the Rust compiler some more in 2018}
	\item \href{https://blog.mozilla.org/nnethercote/2018/11/06/how-to-speed-up-the-rust-compiler-in-2018-nll-edition/}{\Square Nethercote: How to speed up the Rust compiler in 2018: NLL edition}
	\item \href{https://blog.mozilla.org/nnethercote/2018/05/17/the-rust-compiler-is-getting-faster/}{\Square Nethercote: The Rust compiler is getting faster}
	\item \href{https://blog.mozilla.org/nnethercote/2019/07/25/the-rust-compiler-is-still-getting-faster/}{\Square Nethercote: The Rust compiler is still getting faster}
	\item \href{https://blog.mozilla.org/nnethercote/2019/07/17/how-to-speed-up-the-rust-compiler-in-2019/}{\Square Nethercote: How to speed up the Rust compiler in 2019}
	\item \href{https://blog.mozilla.org/nnethercote/2019/10/11/how-to-speed-up-the-rust-compiler-some-more-in-2019/}{\Square Nethercote: How to speed up the Rust compiler some more in 2019}
	\item \href{https://blog.mozilla.org/nnethercote/2019/12/11/how-to-speed-up-the-rust-compiler-one-last-time-in-2019/}{\Square Nethercote: How to speed up the Rust compiler one last time in 2019}
	\item \href{https://blog.mozilla.org/nnethercote/2020/04/24/how-to-speed-up-the-rust-compiler-in-2020/}{\Square Nethercote: How to speed up the Rust compiler in 2020}
	\item \href{https://blog.mozilla.org/nnethercote/2020/08/05/how-to-speed-up-the-rust-compiler-some-more-in-2020/}{\Square Nethercote: How to speed up the Rust compiler some more in 2020}
	\item \href{https://blog.mozilla.org/nnethercote/2020/09/08/how-to-speed-up-the-rust-compiler-one-last-time/}{\Square Nethercote: How to speed up the Rust compiler one last time}
	\item \href{https://pingcap.com/blog/rust-compilation-model-calamity}{\Square PingCAP Blog: The Rust Compilation Model Calamity}
	\item \href{https://pingcap.com/blog/generics-and-compile-time-in-rust}{\Square PingCAP Blog: Generics and Compile-Time in Rust}
	\item \href{https://pingcap.com/blog/rust-huge-compilation-units}{\Square PingCAP Blog: Rust's Huge Compilation Units}
	\item \href{https://pingcap.com/blog/reasons-rust-compiles-slowly}{\Square PingCAP Blog: A Few More Reasons Rust Compiles Slowly}
	\end{itemize}

\item Miscellaneous
	\begin{itemize}[noitemsep]
	\item \href{https://www.youtube.com/watch?v=S2dK5lLFD_0}{\Square Youtube: Making Fast Incremental Compiler for Huge Codebase - Michał Bartkowiak - code::dive 2019}
	\item \href{https://www.youtube.com/watch?v=JbS8a-Ba0Ck}{\Square Youtube: Starting with Semantics - Sylvan Clebsch - PLISS 2019}
	\item \href{https://www.youtube.com/watch?v=mt6pIpt5Wk0}{\Square Youtube: Polyhedral Compilation as a Design Pattern for Compilers (1/2) - Albert Cohen - PLISS 2019}
	\item \href{https://www.youtube.com/watch?v=3TNT5rFVTUY}{\Square Youtube: Polyhedral Compilation as a Design Pattern for Compilers (2/2) - Albert Cohen - PLISS 2019}
	\item \href{https://www.youtube.com/watch?v=yvlhwZgUPG0}{\Square Youtube: First-Class Continuations: What and Why - Arjun Guha}
	\item \href{https://www.youtube.com/watch?v=n_GhkL8GDAk}{\Square Youtube: Implementing First-Class Continuations by Source to Source Translation - Arjun Guha - PLISS 2019}
	\item \href{https://www.youtube.com/watch?v=Lr4cMmaJHrg}{\Square Youtube: Static Program Analysis (part 1/2) - Anders Møller - PLISS 2019}
	\item \href{https://www.youtube.com/watch?v=6QQSIIvH-F0}{\Square Youtube: Static Program Analysis (part 2/2) - Anders Møller - PLISS 2019}
	\end{itemize}

\end{itemize}

\end{document}
